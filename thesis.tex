% Created 2019-01-23 Wed 12:55
% Intended LaTeX compiler: pdflatex
\documentclass[12pt, article]{article}
\usepackage[utf8]{inputenc}
\usepackage[T1]{fontenc}
\usepackage{graphicx}
\usepackage{grffile}
\usepackage{longtable}
\usepackage{wrapfig}
\usepackage{rotating}
\usepackage[normalem]{ulem}
\usepackage{amsmath}
\usepackage{textcomp}
\usepackage{amssymb}
\usepackage{capt-of}
\usepackage{hyperref}
\usepackage[a4paper, left=5cm,right=2cm,top=2cm,bottom=2cm]{geometry}
\usepackage{setspace}
\usepackage{caption}
\onehalfspacing
\usepackage[official]{eurosym}
\usepackage{amsmath}
\usepackage{amssymb}
\usepackage[backend=biber, style=apa]{biblatex}
\addbibresource{~/uni/ma-thesis/bibliography/references.bib}
\usepackage[numbib,notlof,notlot,nottoc]{tocbibind}
\pagenumbering{Roman}
\author{Tobias Richter}
\date{\today}
\title{Reinforcement Learning Portfolio Optimization of Electric Vehicle Virtual Power Plants}
\hypersetup{
 pdfauthor={Tobias Richter},
 pdftitle={Reinforcement Learning Portfolio Optimization of Electric Vehicle Virtual Power Plants},
 pdfkeywords={},
 pdfsubject={},
 pdfcreator={Emacs 26.1 (Org mode 9.2)},
 pdflang={English}}
\begin{document}

\begin{titlepage}
    \begin{center}
        \vspace*{1cm}

        \Large
        \textbf{Reinforcement Learning Portfolio Optimization of Electric Vehicle Virtual Power Plants}

        \vspace{1.5cm}
        Master Thesis

        \vspace{8.0cm}

        \large
        \textbf{Author}: Tobias Richter\\
        \large
        \textbf{Supervisor}: Prof. Dr. Wolfgang Ketter

        \vspace{1cm}
        \large
        Department of Information Systems for Sustainable Society\\
        Faculty of Management, Economics and Social Sciences\\
        University of Cologne\\

        \vspace{1cm}
        October, 2018

    \end{center}
\end{titlepage}
\setcounter{page}{2}

\tableofcontents
\clearpage
\pagenumbering{arabic}

\section{Introduction (10\%)}
\label{sec:org6c2d102}
\subsection{Research Motivation}
\label{sec:orgc11755b}
\begin{itemize}
\item \cite{lopes11_integ_elect_vehic_elect_power_system}
\end{itemize}
\subsection{Research Question}
\label{sec:org09462f1}
\subsection{Relevance}
\label{sec:org068eb29}

\section{Related Literature (10\%)}
\label{sec:org15f0709}
\subsection{Smart Charging and Balancing the Electric Grid with EV Fleets}
\label{sec:org9ab0c43}
The increasing penetration of EVs has a substantial effect on electricity consumption
patterns. During charging periods, power flows and grid losses increase
considerably and challenge the grid. Operators have to reinforce the grid
to ensure that transformers and substations do not get overloaded
\parencite{sioshansi12_impac_elect_tarif_plug_in,lopes11_integ_elect_vehic_elect_power_system}.
Loading multiple EVs in the same neighbourhood, or worse, whole EV fleets at
once, stress the grid. In these cases, even brown- or blackouts are possible
\parencite{kim12_carbit}. Despite these challenges, it is possible to postpone the
physical reinforcement by adopting smart charging strategies. In smart charging,
EVs get charged when the grid is less congested to achieve more grid stability.
Smart charging reduces peaks in electricity demand, called \emph{Peak Cutting} and
complement the grid in times of low demand, called \emph{Valley Filling}. Smart
charging has been researched thoroughly in the IS literature.

\textcite{valogianni14_effec_manag_elect_vehic_storag} find that using intelligent
agents to schedule EV charging, substantially reshapes the energy demand and
reduces peak demand without violating individual household preferences. Moreover,
they show that the proposed smart charging behaviour reduced average energy
prices and thus economically benefit households. In another study
\textcite{kara15_estim_benef_elect_vehic_smart} investigate the effect of smart
charging on public charging stations in California. Controlling for
arrival and departure times, the authors present beneficial results for the
distribution system operator (DSO) and the owners of EVs. A price
reduction in energy bills and a peak load reduction could be determined.
An extension of the smart charging concept is Vehicle-to-Grid (V2G). When
equipped with V2G devices, EVs can discharge their batteries back into the grid.
Several authors conduct research on this technology in respect to grid stabilization
effects and arbitrage possibilities.
\textcite{schill11_elect_vehic_imper_elect_market} find that EVs can be beneficial
for average consumer electricity prices when the EVs can be used as storage.
Excess EV battery capacity can be used to charge in off-peak hours and discharge
in peak hours, when the prices are higher. These arbitrage possibilities,
reverses welfare effects of generators and increases general overall welfare and
consumer surplus. \textcite{tomic07_using_fleet_elect_drive_vehic_grid_suppor}
show that the arbitrage opportunities are especially prominent when a high
variability in electricity prices on the target electricity market exists. The
authors state that short intervals between the contract of sale and the physical
delivery of electricity increase arbitrage benefits. Consequently ancillary service
markets, like frequency control and operating reserve markets are attractive for
smart charging.

\textcite{peterson10_econom_using_plug_in_hybrid} investigate energy arbitrage
profitability with V2G in the light of battery depreciation costs in the US.
Their results indicate that large-scale use of EV batteries for grid storage
does not yield enough profits to incentivize owners to participate in V2G
activities. Considering battery depreciation cost they arrive at an annual
profit of only 6\$ - 72\$ per EV.
\textcite{brandt17_evaluat_busin_model_vehic_grid_integ} evaluated a business
model for parking garage operators operating on the German frequency regulation
market. When taking infrastructure costs and battery depreciation costs into
account they concluded that the proposed vehicle-grid integration is not
profitable. Even with generous assumptions about EV adoption rates in Germany
and altered auction mechanisms they arrived at negative profits.
\parencite{kahlen17_fleet} used EV fleets to offer balancing services to the grid.
Evaluating the impact of V2G in their model the authors come to the conclusion
that V2G would only be profitable if reserve power prices would be twice as
high. Given the results from the aforementioned studies, we decided not to
include V2G into our model, since expected profits are, at most, marginal.

To maximize profits, it is essential for market participants to develop good
bidding strategies. Successful bidding strategies to jointly participate in
multiple markets have been developed e.g. by
\textcite{mashhour11_biddin_strat_virtual_power_plant_2}. The authors use VPPs of
stationary battery storage to participate in the spinning reserve market and
the day-ahead market at the same time. They developed a non-equilibrium model,
which solves the presented mixed-integer program with Genetic Programming (GP).
Contrarily, we use a model-free RL agent that learns an optimal policy (i.e.
trading strategy) from actions it takes in the environment (i.e. bidding on
electricity markets). Using a model-free approach is especially beneficial for
us, since additional unknown variables and constraints (i.e. customer mobility
demand), makes it very hard to formulate a mathematical model.

Similar research to \textcite{mashhour11_biddin_strat_virtual_power_plant_2} has
been done by \textcite{he16_optim_biddin_strat_batter_storag}. The authors
additionally incorporate battery life cycle in their profit maximization model,
which proves to be a decisive factor. In contrast to the authors, we jointly
participated in the secondary operating reserve and spot market with the
\emph{non-stationary} storage of EV batteries. Because shared EVs have to satisfy
mobility demands, they have to be charged nonetheless, which allows us to safely
exclude battery deprecation from our model. Further, we choose the intraday
continuous market over the day-ahead market, as it has the lowest reaction time
of the spot markets, and thus offers potentially higher profits
\parencite{tomic07_using_fleet_elect_drive_vehic_grid_suppor}.



Previous studies often make the assumption that car owners or households can
directly trade on electricity markets. In reality, this is not possible due to
minimum capacity requirements of the markets. For example, the German secondary
reserve market has a 1 MW minimum trading capacity, while the maximum battery
capacity of i.e. a \emph{Smart ForTwo Electric} is 16.50 kWh.

\textcite{ketter13_power_tac} introduced the notion of electricity brokers,
intelligent agents that act on behalf of a group of individuals or households to
participate on electricity markets.
\textcite{brandt17_evaluat_busin_model_vehic_grid_integ} and
\textcite{kahlen14_balan_with_elect_vehic} successfully showed in simulations that
electricity brokers can overcome the capacity issues by aggregating distributed
electricity sources.

Carsharing providers which manage large EV fleets, can use their EVs as VPPs to
participate on electricity markets. We look at the concept of free float
carsharing, an approach which offers more flexibility to its users, saves
resources and reduces carbon emissions
\parencite{firnkorn15_free_float_elect_carsh_fleet_smart_cities}. In most previous
studies concerning using EVs for electricity trading, it was assumed that trips
are fixed and known in advance. The free float concept adds uncertainty and
nondeterministic behavior, as cars can be picked up and parked everywhere and
billing is done by the minute. This makes predictions about the where and when
of a car rental a complex issue. \textcite{wagner16_in_free_float} address this
problem by taking Points of Interests from Google Maps as an additional
predictor.

\textcite{tomic07_using_fleet_elect_drive_vehic_grid_suppor,kahlen17_fleet} showed
that is possible to use free floating carsharing fleets as VPPs to profitably
offer balancing services to the grid. The authors also showed that with a
similar approach, carsharing companies can participate on day-ahead markets for
arbitrage purposes \parencite{kahlen18_elect_vehic_virtual_power_plant_dilem}. A
central dilemma within this research is to decide whether an EV should be
committed to being used as a VPP or to be free for rent. Rental
profits are considerably higher than profits to be made from electricity
trading.

Another central problem is that offering capacity to the grid, which
you can not provide, results in heavy penalties, which should be avoided at all
costs. To address this issue, the authors make use of asymmetric objective
functions that heavily penalize committing an EV to a VPP, when it would have
been rented otherwise. Therefore only very conservative estimations and
commitments of available overall capacity to be traded on the markets are made.
This results in a high amount of foregone profits when bidding on the balancing
market. \textcite{kahlen15_aggreg_elect_cars_sustain_virtual_power_plant} state
that in 42\% to 80\% of the time EVs are \emph{not} committed to a VPP when it would have been
profitable (i.e. the EV has not been rented out).

We are proposing a solution, in which the EV fleet participates on the balancing
market and intraday market simultaneously. With this approach we aim to align
the potentially higher profits on the balancing markets with the more accurate
capacity estimations, which can be made on intraday markets (because time
between commitment and delivery is smaller). We follow
\textcite{kahlen15_aggreg_elect_cars_sustain_virtual_power_plant} with this
approach, who also propose a combination of multiple markets in future work on
this topic.

\subsection{Reinforcement Learning in Smart Grids}
\label{sec:orgf782ada}

Previous research showed that intelligent agents equipped with Reinforcement
Learning methods can successfully take action in the smart grid.
\textcite{reddy11_strat,reddy11_learn_behav_multip_auton_agent} conducted
research, in which autonomous broker agents \parencite{ketter13_power_tac} learn
their strategies using RL. \textcite{peters13_reinf_learn_approac_to_auton} build
on that work and further enhance the method, by learning over larger state
spaces to accommodate arbitrary economic signals. This is especially beneficial
in smart markets, because the markets structures might change in the future and
intelligent agents should adapt to a variety of market structures and
conditions.

\parencite{vazquez-canteli19_reinf_learn_deman_respon}


\textcite{valogianni14_effec_manag_elect_vehic_storag} adopt RL methods to learn
electricity consumption behavior of households. The authors implement these
methods in intelligent agents to smart charge EVs more effectively.
\textcite{vandael15_reinf_learn_heuris_ev_fleet} use RL to learn collective EV
fleet charging behavior to profitably purchase electricity on the day-ahead
market. We consider RL a perfect fit for the design of our proposed intelligent
agent, especially as a solution for our Research Question 2. When dynamically
optimizing the VPP portfolio composition of the fleet, there is no historical
data available to train a model. Using RL and a reward function that maximizes
the overall profitability of the fleet, the agent can learn from its environment
with unknown dynamics and take a certain set of actions. The agent can consider
different states (e.g. current and forecasted rental demand levels and
electricity prices) to take actions (e.g. allocate battery capacity to different
types of VPPs) that maximizes the reward function.

\section{Theoretical Background (10\%)}
\label{sec:orgbbbf773}
\subsection{Electricity Markets}
\label{sec:org3343b91}
\subsubsection{Balancing Market}
\label{sec:org6dcd76c}
\subsubsection{Spot Market}
\label{sec:org8edb461}
\subsection{Reinforcement Learning}
\label{sec:org5303750}
\subsubsection{Notation}
\label{sec:org4fcd4e4}
The input to the network \(x \in \mathbb{R}^D\) is fed to the first residual layer to get the activation \(y = x + \sigma(w x + b) \in \mathbb{R}^D\) with \(w \in \mathbb{R}^{D \times D}\), and \(b \in \mathbb{R}^D\) the weights and bias of the layer.
\subsubsection{Markow Decision Processes}
\label{sec:orga8dcaef}
\subsubsection{Q-Learning}
\label{sec:orgd9d527c}
\subsubsection{Function Approximation}
\label{sec:orgb96a454}
\subsubsection{Exploitation-Exploration Tradeoff}
\label{sec:orgdfd7273}
\subsubsection{Deep Reinforcement Learning}
\label{sec:org22e561f}
\section{Empirical Setting / Data (10\%)}
\label{sec:org5221bf0}
\subsection{Carsharing Fleets of Electric Vehicles}
\label{sec:orga05d2fb}
\subsubsection{Raw Data}
\label{sec:org0d0f5a4}
The dataset consists of 500 EVs in Stuttgart. As displayed in Table
\ref{car2go-sample-data}, the data contain spatio-temporal attributes, such as
timestamp, coordinates, and address of the EVs. Additionally, status attributes
of the interior and exterior are given, the relative state of charge and
information whether the EV is plugged into one of the 200 charging stations in
Stuttgart.

\begin{longtable}{l|ccccc}
\caption{Raw Car2Go Trip Data from Stuttgart \label{car2go-sample-data}}
\\
\hline
\hline
Number Plate & Latitude & Longitude & Street & Zip Code & Engine Type\\
\hline
\endfirsthead
\multicolumn{6}{l}{Continued from previous page} \\
\hline

Number Plate & Latitude & Longitude & Street & Zip Code & Engine Type \\

\hline
\endhead
\hline\multicolumn{6}{r}{Continued on next page} \\
\endfoot
\endlastfoot
\hline
S-GO2471 & 9.19121 & 48.68895 & Parkplatz Flughafen & 70692 & electric\\
S-GO2471 & 9.15922 & 48.78848 & Salzmannweg 3 & 70192 & electric\\
S-GO2471 & 9.17496 & 48.74928 & Felix-Dahn-Str.45 & 70597 & electric\\
S-GO2471 & 9.17496 & 48.74928 & Felix-Dahn-Str.45 & 70597 & electric\\
S-GO2471 & 9.17496 & 48.74928 & Felix-Dahn-Str.45 & 70597 & electric\\
\hline
Number Plate & Interior & Exterior & Timestamp & Charging & State of Charge\\
\hline
S-GO2471 & good & good & 22.12.2017 20:10 & no & 94\\
S-GO2471 & good & good & 24.12.2017 23:05 & no & 72\\
S-GO2471 & good & good & 26.12.2017 00:40 & yes & 81\\
S-GO2471 & good & good & 26.12.2017 00:45 & yes & 83\\
S-GO2471 & good & good & 26.12.2017 00:50 & yes & 84\\
\hline
\hline
\end{longtable}
\subsubsection{Preprocessing Steps}
\label{sec:org2d060bb}
\subsection{Electricity Markets Data}
\label{sec:orgb060943}
\subsubsection{Secondary Operating Reserve Market}
\label{sec:orged2724f}
\subsubsection{Intraday Continuous Spot Market}
\label{sec:orgcd82e67}

\section{Model: FleetRL (20\%)}
\label{sec:orgb3589e8}
\subsection{Information Assumptions}
\label{sec:org67ba226}
\subsection{Mobility Demand \& Clearing Price Prediction}
\label{sec:orga911f2f}
\subsection{Reinforcement Learning Approach}
\label{sec:org4c58fc2}
\subsection{Bidding Strategy}
\label{sec:org1cc0316}

\section{Evaluation (30\%)}
\label{sec:org75759f8}
\subsection{Event-based Simulation}
\label{sec:orga02b142}
\subsection{Benchmark: Ad-hoc Strategies}
\label{sec:org33d7461}
\subsection{FleetRL}
\label{sec:orgce64637}
\subsection{Sensitivity Analysis: Prediction Accuracy}
\label{sec:org0ce6e5f}
\subsection{Sensitivity Analysis: Infrastructure Changes}
\label{sec:orgb87950b}
\subsection{Sensitivity Analysis: Bidding Strategy}
\label{sec:org4e0919a}
\section{Discussion (5\%)}
\label{sec:org1c1e36c}
\subsection{Generalizability}
\label{sec:orged32c61}
\subsection{Future Electricity Landscape}
\label{sec:orgd0125af}
\subsection{Limitations}
\label{sec:org786b142}
\section{Conclusion (5\%)}
\label{sec:org97c0269}
\subsection{Contribution}
\label{sec:orgf934762}
\subsection{Future Research}
\label{sec:org1fc09e8}



\clearpage
\printbibliography
\end{document}
